\documentclass[conference]{IEEEtran}
\IEEEoverridecommandlockouts
% The preceding line is only needed to identify funding in the first footnote. If that is unneeded, please comment it out.
\usepackage{amsthm}
\usepackage{longtable}
\usepackage{upgreek}
\usepackage{amsmath}
\usepackage{mathtools}
\usepackage{graphicx} 
\usepackage{gensymb}
\usepackage{hyperref}
\usepackage{amssymb}
\usepackage{amsmath}
\usepackage{dsfont}
%\usepackage{amsfonts}
%\usepackage{bbold}
\newcommand{\rEarth}{R_{\oplus}}
\newcommand{\viiva}{\mathop{\Bigg/}}
\newcommand{\sij}[3]{\viiva\limits_{\hspace*{-5mm}{#1}}^{\hspace*{5mm}{#2}}{#3}}
\newcommand{\R}{\mathbb{R}}
\newcommand{\N}{\mathbb{N}}
\newcommand{\B}[1]{\mathbf{#1}}
\newtheorem{thm}{Theorem} 
\newtheorem{thm1}{Theorem}
\newtheorem{thm2}{Theorem}
\newtheorem{thm3}{Theorem}
\theoremstyle{definition}
\newtheorem{lemma}[thm1]{Lemma}
\newtheorem{remark}[thm2]{Remark}
\newtheorem{defn}[thm3]{Definition}
\theoremstyle{plain}
\newtheorem{thm4}{Theorem}
\newtheorem{thm5}{Theorem}
\newtheorem{thm6}{Theorem}
\newtheorem{thm7}{Theorem} 
\newtheorem{prop}[thm4]{Proposition}
\newtheorem{kor}[thm5]{Corollary}
\newtheorem{conjecture}[thm6]{Conjecture}
\newtheorem{approximation}[thm7]{Approximation}
\usepackage{amsmath,amsfonts}
\usepackage{algorithmic}
\usepackage{array}
\usepackage[caption=false,font=normalsize]{subfig}
\usepackage{textcomp}
\usepackage{stfloats}
\usepackage{url}
\usepackage{verbatim}
\usepackage{graphicx}
\hyphenation{op-tical net-works semi-conduc-tor IEEE-Xplore}
\def\BibTeX{{\rm B\kern-.05em{\sc i\kern-.025em b}\kern-.08em
    T\kern-.1667em\lower.7ex\hbox{E}\kern-.125emX}}
\usepackage{balance}

\def\BibTeX{{\rm B\kern-.05em{\sc i\kern-.025em b}\kern-.08em
    T\kern-.1667em\lower.7ex\hbox{E}\kern-.125emX}}
\begin{document}

\title{Stochastic geometry analysis of a narrow-beam LEO uplink with mixed Gaussian shadowing}

\author{\IEEEauthorblockN{1\textsuperscript{st} Ilari Angervuori}
\IEEEauthorblockA{\textit{Department of Electrical Engineering} \\
\textit{Aalto University}\\
Espoo, Finland \\
ilari.Angervuori@aalto.fi}
\and
\IEEEauthorblockN{3\textsuperscript{rd} Risto Wichman}
\IEEEauthorblockA{\textit{Department of Electrical Engineering} \\
\textit{Aalto University}\\
Espoo, Finland \\
risto.wichman@aalto.fi}
}
\maketitle
         \begin{abstract}
         \end{abstract}

         % Note that keywords are not normally used for peerreview papers.
         \begin{IEEEkeywords}
           LEO, SIR meta distribution, Nakagami fading
         \end{IEEEkeywords}


         \section{Introduction}

         
         \begin{table}
           \begin{center}
             \begin{tabular}{| c | p{7cm}  |}
               \hline
               \multicolumn{2}{|c|}{Glossary of principal symbols} \\
               \hline
               Symbol& Explanation 
               \\ 
               \hline
               $h$ & Altitude of the SBSs. \\
               $\epsilon$ & Elevation angle of the SBSs. \\
               $G[\cdot]$ & The SBS antenna gain.\\
               $\varphi_{\text{RX}}$ & Width of the SBSs $3$ dB gain. \\
               $\Theta \subset E $ & Poisson p.p. on the earth's surface $E \subset \R^3$. \\
               $\Phi \subset \R^2$ & Poisson p.p. on the plane. \\
               $x_0$ & Nearest point to the origin in $\Phi$.  \\
               $\lambda$ & Density parameter of $\Phi$ and $\Theta$. \\
               $\kappa$ & Parameter that reflects the approximate mean number of UEs inside a SBS's $3$ dB footprint;  $\kappa = h^2\pi \lambda \varphi_{\textup{RX}}^2/\sin^4(\epsilon)$. \\
               ${\tilde{\kappa}}$ &  $\kappa/\log(2)$.\\
               $g_x$ &   Gamma-distributed random fading gain of a transmitter $x$.     \\
               $\theta$ & SIR or SINR threshold for a successful transmission.\\
               $I$ & Interference at the typical SBS in the plane model.\\
               $S$ & The signal power of the served UE at the typical SBS in the plane model.\\
               $\mathring{I}$ & Interference at the typical SBS in the spherical model.\\
               $\mathring{S}$ & The signal power of the served UE at the typical SBS in the spherical model.\\ 
               $\hat{d}_{h,\epsilon}$ & The distance between the SBS and the center of the footprint in the plane model.\\
               $d_{0}$ & Normalizing distance.  \\                        
               \hline
             \end{tabular}
           \end{center}
         \end{table}   

         
         \section{Analysis}
         
         \subsection{Gain process}

         The path loss law is Gaussian, a exponentially decaying function, representing the antenna gain of a narrow-beam LEO satellite. The path loss is a function of the distance $r$ from the origin, 
          \begin{equation}
            G(r) = 2^{-(D_{h,\epsilon}r)^2/\varphi^2_{\text{RX}}},
          \end{equation}
          where $\varphi_{\text{RX}}$ is the width of the $3$ dB gain and $D_{h,\epsilon}=\sin^2(\epsilon)/h$ is a scaling constant. We will come back to the interpetation of these parameters in the LEO satellite context in Section XXX. 
          

         Given a homogeneous PPP $\Phi \subset \R^2$ of density $\lambda$, define the shadowed gain process (GP)
         \begin{equation}
           \label{eq:gainprocess}
           \mathcal{G} \triangleq \left\{x \in \Phi : H_x G(\|x\|) \right\},
          \end{equation}
          where $\{H_x\}$ are i.i.d. shadowing variables, possibly degenerate, and $\|x\|$ is the Euclidean distance. We use the value $\varphi_{\text{RX}} = 1.6 \degree$ according to the International Telecommunication Union Recommendations (ITU-R) \cite{ITURS1528}. The GP is a \textit{projection process} and, as such, a nonhomogeneous PPP [CITE].

          The density of the GP has the following connection to the fading distribution:

            \begin{prop}[Density of the GP]
              Let $F_H(\cdot)$ be the complementary cdf (ccdf) of $H$. The density of $\mathcal{G}$ is given by
              \begin{equation}
                \label{eq:GPdensity}
                \lambda_{\mathcal{G}}(t)= \tilde{\kappa}F_H(t)/t, \text{ \text{for} }t>0,
              \end{equation}
              where $\tilde{\kappa}= \kappa/\log(2)$ and
              \begin{equation}
                \label{eq:kappa}
                      {\kappa} \triangleq    \pi \lambda   \left(\frac{\varphi_{\textup{RX}}}{D_{h,\epsilon}}\right)^2.
              \end{equation}
              The parameter $\kappa$ has an interpretation as the mean number of UEs inside the $3$ dB footprint of a SBS (see appendix XX).
              \begin{proof}
                Let $f_H(\cdot)$ be the pdf of $H$. We denote $G^{-1}(\cdot)$ as the generalized inverse $G^{-1}(y) = \inf \{x:G(x)<y\}$. By [CITE],
                \begin{align*}
                  &\int_t^{\infty}\lambda_{\mathcal{G}}(y)dy \overset{}{=} \pi \lambda \mathbb{E}\left[ \left(\frac{G^{-1}(t/H)}{D_{h,\epsilon}}\right)^2 \right] \\
                  &=\pi \lambda \int_t^{\infty} \left(-\frac{\varphi_{\text{RX}}\sqrt{-\log(t/h)}}{D_{h,\epsilon} \sqrt{\log(2)}}\right)^2f_H(h) dh  \\
                  &= -\tilde{\kappa} \int_t^{\infty} \log(t/h)f_H(h) dh\\
                  &\overset{(a)}{=} -\tilde{\kappa} \sij{t}{\infty} \log(t/h)F_H(h) - \tilde{\kappa} \int_t^{\infty}\frac{F_H(h)}{h} dh.
                \end{align*}
                In (a), we use the integration by parts. The result follows by derivating w.r.t. $t$ and taking the minus sign---as long as $\int_t^{\infty} \log(t/h)f_H(h) dh$ converges for all $t>0$.
              \end{proof}
            \end{prop}

          The mean and the variance of the total received power are given by
          \begin{align}
            &\mathbb{E}\left(\sum_{x \in \mathcal{G}} x \right) = \int_{0}^{\infty} t\lambda_{\mathcal{G}}(t) dt = \tilde{\kappa} \int_{0}^{\infty}F_H(t) dt \nonumber \\
            &=\tilde{\kappa} \mathbb{E}(H), \\\
            &\text{var}\left(\sum_{x \in \mathcal{G}} x \right) = \int_{0}^{\infty} t^2\lambda_{\mathcal{G}}(t) dt= \tilde{\kappa} \int_0^{\infty}tF_H(t) dt  \nonumber \\
           & \tilde{\kappa} \frac{\text{var}(H) + \mathbb{E}(H)^2}{2} = \tilde{\kappa}  \mathbb{E}[H^2]/2,
          \end{align}
          respectively. 

          Unfortunately, unlike in terrestrial networks with a singular path loss, where the density of the projection process is dependent only on a single moment of $H$, $\lambda_{\mathcal{G}}(t)$ has an explicit pointwise dependence on the ccdf of the fading distribution. 


                      \subsection{Slow-fading distribution}
 While analytically tractable, the defective exponential distribution can capture the mean and the variance of a more complicated fading distribution with high variance, particularly the lognormal distribution, which is a well-established shadowing model in the LEO networks. The shadowing is defined by the mixed exponential RV
             \begin{equation}
               \hat{H}_x \sim
            \begin{cases}
               0, \text{ if } U < 1-b,\\
                \text{Exp}(a), \text{ if } U \geq1- b,              \label{eq:tier2exponential}
            \end{cases}
            \end{equation}
             with $a>0$ and $0<b\leq1$, and $U \sim U(0,1)$ follows the uniform distribution.


            The complementary CDF (CCDF) is given by the defective distribution function
            \begin{equation}
              \label{eq:defexp}
              F_{\hat{H}_{\text{exp}}}(t)=e^{-at}b.
            \end{equation}
            %%where
            %% \begin{align}
            %%   &a=\frac{2 (p_{\text{LoS}}-1) e^{{\mu_{N\text{LoS}}}+\frac{{\sigma_{N\text{LoS}}}^2}{2}}-2 p_{\text{LoS}} e^{{\mu_{\text{LoS}}}+\frac{{\sigma_{\text{LoS}}}^2}{2}}}{(p_{\text{LoS}}-1) e^{2 \left({\mu_{N\text{LoS}}}+{\sigma_{N\text{LoS}}}^2\right)}-p_{\text{LoS}} e^{2 \left({\mu_{\text{LoS}}}+{\sigma_{\text{LoS}}}^2\right)}},\\
            %%   &b=\frac{2 \left(({p_\text{LoS}}-1) e^{{\mu_{\text{N\text{LoS}}}}+\frac{{\sigma_{\text{N\text{LoS}}}}^2}{2}}-{p_\text{LoS}} e^{{\mu_{\text{LoS}}}+\frac{{\sigma_{\text{LoS}}}^2}{2}}\right)^2}{({p_\text{LoS}}-1) e^{2 \left({\mu_{\text{N\text{LoS}}}}+{\sigma_{\text{N\text{LoS}}}}^2\right)}-{p_\text{LoS}} e^{2 \left({\mu_{\text{LoS}}}+{\sigma_{\text{LoS}}}^2\right)}}.              
            %% \end{align} 
            At the probability $1-b$, the signal power maps to $0$: this reflects highly severe shadowing occurrences. 




          \subsection{Total interference and its Laplace transform}

          The Laplace function of the total of the points in $\mathcal{G}$ is defined as $S_{\mathcal{G}} \triangleq \sum_{x \in \mathcal{G}}x$, i.e., the \textit{total received power} scaled by the spatial path loss $\ell(\hat{d}_{h,\epsilon})$ from all UEs in $\mathcal{G}$ at the typical SBS, is given by

          \begin{equation}
            \label{eq:lapdef}
            \mathcal{L}_{S_{\mathcal{G}}}(s)\triangleq \mathbb{E}\left(e^{-s\sum\limits_{x \in \mathcal{G}}x}\right)= \exp\left\{-\int_0^{\infty}(1-e^{-sr}) \lambda_{\mathcal{G}}(r) dr \right\}.
          \end{equation}
          For independent $\{\mathcal{G}_i\}_{i \in \mathcal{\tau}}$, $\mathcal{L}_{\sum_{i \in \mathcal{T}} \mathcal{G}_i}(s) = \prod_{i \in \mathcal{T}} \mathcal{L}_{\mathcal{G}_i}(s)$

          
          \subsubsection{Rayleigh fading case}
          
          We are ready to derive a approximation for the probability of coverage of the UE with the srongest signal.

          \subsection{Signal-to-interference ratio}
          Define the SIR of the $k$\textit{th} strongest signal at the typical SBS as
          \begin{equation}
            \text{SIR}_k \triangleq \frac{x_k}{\sum\limits_{x \in \mathcal{G} \setminus \{x_0\}}x},
          \end{equation}
          $k\in\{1,2,\dots\}$.
            

          \subsection{Gaussian mixture model}
          Let us consider a two-tier $\mathcal{T} = \{\text{LoS},\text{NLoS}\}$ logarithmic Gaussian mixture fading model with the mean and standard deviation $\mu_{i},\sigma_{i}$, $i \in \{\text{LoS},\text{NLoS} \}$, for the line-of-sight and the non-line-of-sight tiers, respectively. Consequently, The power fading RV of a transmitter $x \in \Phi$ is given by
            \begin{equation}
            H_x \sim
            \begin{cases}
               \text{Lognormal}(\mu_{\text{LoS}},\sigma_{\text{LoS}}^2), \text{ if } U<p_{\text{LoS}} \\
               \text{Lognormal}(\mu_{\text{NLoS}},\sigma_{\text{NLoS}}^2)\text{ if } U \geq p_{\text{NLoS}},              \label{eq:tier2lognormal}
            \end{cases}
            \end{equation}
            where $U \sim U(0,1)$ follows the uniform distribution.
            
            The mean of the lognormal RV is given by $ m_i=\mathbb{E}(H|T=i) = \exp \{\mu_i+\sigma_i^2/2\}$.
            



            \subsection{Distribution of the total received power}
             Under the shadowing approximation $\eqref{eq:defexp}$, the total received power can be simply characterized by the gamma distribution.
          \begin{align}
            \label{eq:lapexp}
            &\mathcal{L}_{S_{\mathcal{G}}}(s) \approx \exp\left\{-\mathcal{\kappa}\int_0^{\infty}(1-e^{-sr})\frac{F_{\hat{H}}(t)}{r} dr \right\}, \nonumber \\
            &= \exp\left\{-\mathcal{\kappa}\int_0^{\infty}(1-e^{-sr})\frac{e^{-ra}b}{r} dr \right\}=(1+s/a)^{-b\tilde{\kappa}}, 
          \end{align}
          which is the Laplace transform of the gamma distribution with shape parameter $b\tilde{\kappa}$ and scale parameter $1/a$.


                    
          Let $(\tau_s, s\geq 0)$ be a \textit{gamma subordinator}, that is, a pure-jump increasing Lévy process with the intensity measure $\lambda(r) = e^{-ar}/r$ . Assume $(\tau_s)$ has no drift component, so
          \begin{equation}
            \label{eq:lapsubord}
            \mathbb{E}(\exp\{-t \tau_s\}) = \exp\left\{-s \int_0^{\infty}(1-e^{-t r})\frac{e^{-ar}}{r} dr \right\}.
          \end{equation}
          But \eqref{eq:lapsubord} equals to \eqref{eq:lapexp} for $t=b \kappa$. Let $V_1(\tau_s) \geq V_2(\tau_s)\geq \dots \geq 0 $ denote the jumps of the subordinator $(\tau_s)$ at time $s$. By definition, the sequence
          \begin{equation}
            \left(\frac{V_1(\tau_{b\kappa})}{\tau_{b\kappa}},\frac{V_2(\tau_{b\kappa})}{\tau_{b\kappa}} \dots \right)
          \end{equation}
          is the process of the STIR values of the UEs $\{x_1,x_2 \dots\}$. It has the Poisson-Dirichlet distribution PD$(0, b \kappa)$.
                    

          In an interference-only channel, $W=0$, the coverage probability \eqref{eq:probofcov} has the analytical expression
          \begin{equation}
p_c(\theta) =  \theta^{-\tilde{\kappa}b} {_2F_1}\left(\tilde{\kappa}b, \tilde{\kappa}b;\tilde{\kappa}b + 1; -1/\theta \right),
          \end{equation}
          where $_2F_1(\cdot)$ is the hypergeometric function. 

          And the probability density function (PDF) in the region $z \geq 1$,
          \begin{equation}
            f_{(1)}(z) = p_c'(z)= \tilde{\kappa} b  \left(\frac{1}{z }+1\right)^{- \tilde{\kappa}b } z^{- \tilde{\kappa}b -1}.
          \end{equation}

          The coverage probability does not depend on the scale parameter $a$.
           
          

                    

          

          
          \begin{prop}
            The density of the nth factorial moment measure of the STIR process at the NB LEO with the Gaussian antenna beam is given by
            \begin{align}
              \label{eq:factorialmoment}
              \mu'^{(n)}(t_1',\dots,t'_n) = (\tilde{\kappa}b)^n\prod_{j=1}^n{t'}_{j}^{-1}\left(1- \sum_{j=1}^nt'_j \right)^{\tilde{\kappa}b-1},       
            \end{align}
            whenever $t_1>\dots >t_n$ and $\sum_{i=1}^n t_i <1$, and $0$ otherwise.
          \end{prop}
          To study the the joint distribution of order statistics of the STINR process, we define some auxillary functions. We write for $i\geq 1$
          \begin{align}
            \label{eq:auxillary}
                  &{\mu'}_k^{(k+i)}(z'_1,\dots,z'_k) \nonumber \\
                  &= \int_{z'_k}^1 \dots \int_{z'_k}^1 {\mu'}^{(k+i)}(z'_1,\dots,z'_k,\zeta'_1,\dots,\zeta'_i) d\zeta'_1 \dots \zeta'_i
          \end{align}
           if $\sum_{i=1}^k t_i + it_k <0$ and $0$ otherwise.

          
          The joint probability density of the vector of $k$ strongest values of the STIR process $(\mathsf{Z}'_{(1)}, \dots, \mathsf{Z}'_{(n)})$ is equal to
          \begin{equation}
            \label{eq:jointprobability}
            f'_{(k)}(z'_1,\dots,z'_k)= \sum^{i_{\text{max}}}_{i=0}\frac{(-1)^i}{i!}{\mu'}_k^{(k+i)}(z'_1,\dots,z'_k),
          \end{equation}x
          where $i_{\text{max}}<1/z'_k-k$, is the maximum index of the non-zero terms of the expansion.e
          The $k$-coverage probability that the first $k$ strongest signals reach the threshold $\tau' =\tau/(1+\tau)$ is given by
          \begin{equation}
      \mathcal{P}^{(k)}(\tau) \triangleq  \int_{\tau'}^1\dots \int_{\tau'}^1 f'_{(k)}({z'_1},\dots,{z'_k})dz'_1 \dots d{z'_k}.
          \end{equation}
          
          The joint pdf of the SIR process can by obtained from $\eqref{eq:jointprobability}$ by
          \begin{equation}
            \label{eq:SIRjointprobability}
            \mu^{(n)}(t_1,\dots,t_n) = \prod_{j=1}^n\frac{1}{(1+t_j)^2}\mu'^{(n)}\left(\frac{t_1}{1+t_1},\dots,\frac{t_n}{1+t_n}\right).
          \end{equation}

          Combining \eqref{eq:SIRjointprobability}, \eqref{eq:jointprobability} and \eqref{eq:factorialmoment}, we get the PDF for the SIR of the strongest signal for $z\geq 1$:
          
          \begin{equation}
            f_{(1)}(z) = \frac {b\kappa\left({z + 1} \right)^{-b\tilde{\kappa}}} {z}.
          \end{equation}


          We may calculate a lower bounds for the first two moments for the SIR:



          \begin{align}
            &\mathbb{E}(\text{SIR}_1)  \geq\int_{1}^{\infty}f_{(1)}(z)zdz=\frac{b \tilde{\kappa}  2^{1-b \kappa }}{b \tilde{\kappa} -1}, \tilde{\kappa}b >1, \\
            &\mathbb{E}(\text{SIR}^2_1) \geq \int_{1}^{\infty}f_{(1)}(z)z^2dz = \frac{(b \tilde{\kappa}) ^2 2^{1-b \tilde{\kappa} }}{(b\tilde{\kappa} - 2)  (b\tilde{\kappa} - 1)},\tilde{\kappa}b >2.
          \end{align}
          The mean and the second moment diverge for $\tilde{\kappa}b\leq 1$ and $\tilde{\kappa}b\leq 2$, respectively. This implies that the variance of SIR$_1$ is undefined for $\tilde{\kappa}b \leq 1$ and infinite for $1 <\tilde{\kappa}b \leq 2$. The insight is crucial because it reflects that, at the optimal \textit{average} performance $\tilde{\kappa}b = 1$, the user experience of the link quality varies significantly. Recall \eqref{eq:kappa}: $\tilde{\kappa}b$ depends on the shadowing (small $b$ corresponds to severe shadowing), UE density, antenna gain width, altitude and elevation angle. We will demonstrate that signal cancellation and combination can improve user fairness without degrading the link quality.



          
          Finally, we arrive to the ICSC-SIR probability of coverage.
          \begin{prop}
            \begin{align}
              &\mathcal{P}^{(\mathbf{K},k)}(\tau) \nonumber\\
              &= \sum^{i_{\text{max}}}_{i=0} \int_{\epsilon}^1 \dots \int_{\epsilon}^1\frac{(-1)^i}{i!}\mathds{1}\left(\tau' \sum_{i \in [k] \setminus  \mathbf{K}}z'_i+ \sum_{i\in \mathbf{K}} z'_i>\tau' \right) \nonumber \\
              &\hspace{0.7cm}\times \mathds{1}(z'_1>\dots>z'_k){\mu'}_k^{(k+i)}(z'_1,\dots,z'_k) d z'_1 \dots z'_k,
            \end{align}
            where $\tau' = \tau/(1+\tau)$ and the sum is bounded by $i_{\text{max}}< 1/\epsilon-k$. 
          \end{prop}

          
          
          
        

          
          
          
          \section{Numerical results}
          
          
          
          

          
         \bibliographystyle{IEEEtran}
         %\bibliography{IEEEabrv, bib}
         \bibliography{IEEEabrv,source}
             
             
\end{document}
