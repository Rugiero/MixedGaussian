\documentclass[conference]{IEEEtran}
\IEEEoverridecommandlockouts
% The preceding line is only needed to identify funding in the first footnote. If that is unneeded, please comment it out.
\usepackage{amsthm}
\usepackage{longtable}
\usepackage{upgreek}
\usepackage{amsmath}
\usepackage{mathtools}
\usepackage{graphicx} 
\usepackage{gensymb}
\usepackage{hyperref}
\usepackage{amssymb}
\usepackage{amsmath}
\usepackage{dsfont}
% \usepackage{amsfonts}
 % \usepackage{bbold}
\newcommand{\rEarth}{R_{\oplus}}
\newcommand{\viiva}{\mathop{\Bigg/}}
\newcommand{\sij}[3]{\viiva\limits_{\hspace*{-5mm}{#1}}^{\hspace*{5mm}{#2}}{#3}}
\newcommand{\R}{\mathbb{R}}
\newcommand{\N}{\mathbb{N}}
\newcommand{\B}[1]{\mathbf{#1}}


\newtheorem{thm}{Theorem} 
\newtheorem{thm1}{Theorem}
\newtheorem{thm2}{Theorem}
\newtheorem{thm3}{Theorem}
\theoremstyle{definition}
\newtheorem{lemma}[thm1]{Lemma}
\newtheorem{remark}[thm2]{Remark}
\newtheorem{defn}[thm3]{Definition}
\theoremstyle{plain}
\newtheorem{thm4}{Theorem}
\newtheorem{thm5}{Theorem}
\newtheorem{thm6}{Theorem}
\newtheorem{thm7}{Theorem} 
\newtheorem{prop}[thm4]{Proposition}
\newtheorem{kor}[thm5]{Corollary}
\newtheorem{conjecture}[thm6]{Conjecture}
\newtheorem{approximation}[thm7]{Approximation}



\usepackage{amsmath,amsfonts}
\usepackage{algorithmic}
\usepackage{array}
\usepackage[caption=false,font=normalsize]{subfig}
\usepackage{textcomp}
\usepackage{stfloats}
\usepackage{url}
\usepackage{verbatim}
\usepackage{graphicx}
\hyphenation{op-tical net-works semi-conduc-tor IEEE-Xplore}
\def\BibTeX{{\rm B\kern-.05em{\sc i\kern-.025em b}\kern-.08em
    T\kern-.1667em\lower.7ex\hbox{E}\kern-.125emX}}
\usepackage{balance}

\def\BibTeX{{\rm B\kern-.05em{\sc i\kern-.025em b}\kern-.08em
    T\kern-.1667em\lower.7ex\hbox{E}\kern-.125emX}}
\begin{document}

\title{Stochastic geometry analysis of a LEO uplink with mixed Gaussian shadowing}

\author{\IEEEauthorblockN{1\textsuperscript{st} Ilari Angervuori}
\IEEEauthorblockA{\textit{Department of Electrical Engineering} \\
\textit{Aalto University}\\
Espoo, Finland \\
ilari.Angervuori@aalto.fi}
\and
\IEEEauthorblockN{3\textsuperscript{rd} Risto Wichman}
\IEEEauthorblockA{\textit{Department of Electrical Engineering} \\
\textit{Aalto University}\\
Espoo, Finland \\
risto.wichman@aalto.fi}
}
\maketitle
         \begin{abstract}
         \end{abstract}

         % Note that keywords are not normally used for peerreview papers.
         \begin{IEEEkeywords}
           LEO, SIR meta distribution, Nakagami fading
         \end{IEEEkeywords}


         \section{Introduction}

         hhihhi
        
          \begin{table}
           \begin{center}
             \begin{tabular}{| c | p{7cm}  |}
               \hline
               \multicolumn{2}{|c|}{Glossary of principal symbols} \\
               \hline
               Symbol& Explanation 
               \\ 
               \hline
               $h$ & Altitude of the SBSs. \\
               $\epsilon$ & Elevation angle of the SBSs. \\
               $G[\cdot]$ & The SBS antenna gain.\\
               $\varphi_{\text{RX}}$ & Width of the SBSs $3$ dB gain. \\
               $\Theta \subset E $ & Poisson p.p. on the earth's surface $E \subset \R^3$. \\
               $\Phi \subset \R^2$ & Poisson p.p. on the plane. \\
                $\mathcal{G} \subset (0,1)$ & A nonhomogeneous Poisson p.p.; the gain process of the approximate signal gains at the typical SBS.  \\
               $x_0$ & Nearest point to the origin in $\Phi$.  \\
               $\lambda$ & Density parameter of $\Phi$ and $\Theta$. \\
               $\kappa$ & Parameter that reflects the approximate mean number of UEs inside a SBS's $3$ dB footprint;  $\kappa = h^2\pi \lambda \varphi_{\textup{RX}}^2/\sin^4(\epsilon)$. \\
               ${\tilde{\kappa}}$ &  $\kappa/\log(2)$.\\
               $g_x$ &   Gamma distributed random fading gain of a transmitter $x$.     \\
               $\theta$ & SIR or SINR threshold for a successful transmission.\\
               $I$ & Interference at the typical SBS in the plane model.\\
               $S$ & The signal power of the served UE at the typical SBS in the plane model.\\
               $\mathring{I}$ & Interference at the typical SBS in the spherical model.\\
               $\mathring{S}$ & The signal power of the served UE at the typical SBS in the spherical model.\\ 
               $\hat{d}_{h,\epsilon}$ & The distance between the SBS and the center of the footprint in the plane model.\\
               $d_{0}$ & Normalizing distance.  \\           
             
               \hline
             \end{tabular}
           \end{center}
         \end{table}   

      
          \section{Analysis}

          \subsection{Gain process}
          Let the constant $D_{h,\epsilon} \triangleq \sin^2(\epsilon)/h$ be the derivative of the function $\|x\| \mapsto \varphi_x $ at $\|x\| =0$. Consequently, $\varphi_x \approx D_{h,\epsilon}\|x\|$ for small $\|x\|$.
          Define the gain process (GP)
          \begin{equation}
            \mathcal{G} = \left\{x \in \Phi : H_x G[D_{h,\epsilon}\|x\|] \right\},
          \end{equation}
          where $\{H_x\}$ are i.i.d. shadowing variables (possibly degenerate), and 
          \begin{equation}
            G[\cdot] = 2^{-(\cdot)^2/\varphi^2_{\text{RX}}}
          \end{equation}
          is the Gaussian antenna pattern $G:[0,\infty) \rightarrow [0,1]$ characterized by the halfwidth of the $3$ dB gain $\varphi_{\text{RX}}$. We use the value $\varphi_{\text{RX}} = 1.6 \degree$ according to the International Telecommunication Union Recommendations (ITU-R) \cite{ITURS1528}. The GP is a \textit{projection process} and, as such, a nonhomogeneous PPP [CITE].

            The density of the GP has the following connection to the fading distribution:

          \begin{prop}[Density of the GP]
            Let $f_H(\cdot)$ and $F_H(\cdot)$ be the pdf and the complementary cdf (ccdf) of $H$, respectively. The density of $\mathcal{G}$ is given by
            \begin{equation}
              \label{eq:GPdensity}
              \lambda_{\mathcal{G}}(t)= \tilde{\kappa}F_H(t)/t
            \end{equation}
            for $t>1$.
            \begin{proof}
              By [CITE],
              \begin{align*}
                &\lambda_{\mathcal{G}}(t) \overset{}{=} \pi \lambda \mathbb{E}\left[\left( G^{-1}[t/H] \right)^2 \right]=\tilde{\kappa}\frac{d}{dt} \int_t^{\infty} G^{-1}[t/y]f_H(y) dy\\
                &  = \tilde{\kappa}\frac{d}{dt} \int_t^{\infty} \log(t/y)f_H(y) dy = \tilde{\kappa} \sij{t}{\infty} \log(t/y)F_H(y)\\ 
                & + \tilde{\kappa} \frac{d}{dt} \int_0^{t}F_H(y)/y dy =  \tilde{\kappa}F_H(t)/t,
              \end{align*}
            as long as $\log(t/y) F_H(y) =0$ as $y \rightarrow 0$ for all $t>0$.  $G^{-1}[\cdot]$ is considered to be the generalized inverse $G^{-1}[\cdot] = \inf \{x:G[x]<y\}$.
              \end{proof}
          \end{prop}

          The mean and the variance of the total received power are given by
          \begin{align}
            &\mathbb{E}\left(\sum_{x \in \mathcal{G}} x \right) = \int_{0}^{\infty} t\lambda_{\mathcal{G}}(t) dt = \tilde{\kappa} \int_{0}^{\infty}F_H(t) dt \nonumber \\
            &=\tilde{\kappa} \mathbb{E}(H), \\\
            &\text{var}\left(\sum_{x \in \mathcal{G}} x \right) = \int_{0}^{\infty} t^2\lambda_{\mathcal{G}}(t) dt= \tilde{\kappa} \int_0^{\infty}tF_H(t) dt  \nonumber \\
           & \tilde{\kappa} \frac{\text{var}(H) + \mathbb{E}(H)^2}{2} = \tilde{\kappa}  \mathbb{E}[H^2]/2,
          \end{align}
          respectively. 

                    Unfortunately, unlike in terrestrial networks with a singular path loss, where the density of the projection process is dependent only on a single moment of $H$, $\lambda_{\mathcal{G}}(t)$ has an explicit pointwise dependence on the ccdf of the fading distribution. However, the density functions are similar for certain fading distributions with matched moments. For example,
          
          \subsection{Explicit fading model}
          The power fading distribution for the mixed Gaussian model is given by




          
         


         \bibliographystyle{IEEEtran}
         %\bibliography{IEEEabrv, bib}
         \bibliography{IEEEabrv,source}
             
             
\end{document}
