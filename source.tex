\documentclass[lettersize,journal]{IEEEtran}
\usepackage{amsmath,amsfonts}
\usepackage{algorithmic}
\usepackage{algorithm}
\usepackage{array}
\usepackage[caption=false,font=normalsize,labelfont=sf,textfont=sf]{subfig}
\usepackage{textcomp}
\usepackage{stfloats}
\usepackage{url}
\usepackage{verbatim}
\usepackage{graphicx}
\usepackage{cite}
\hyphenation{op-tical net-works semi-conduc-tor IEEE-Xplore}
\usepackage{booktabs}
\usepackage{amsmath,amsfonts}
\usepackage{algorithmic}
\usepackage{array}
\usepackage{textcomp}
\usepackage{stfloats}
\usepackage{subcaption}
\usepackage{url}
\usepackage{verbatim}
\usepackage{graphicx}
\usepackage{float}
\usepackage{titlesec}
\usepackage{amsthm}
\usepackage{mathtools}
\usepackage{gensymb}
\usepackage{hyperref}
\usepackage{amssymb}
\usepackage{amsmath}
\usepackage{dsfont}


\newcommand{\rEarth}{R_{\oplus}}
\newcommand{\viiva}{\mathop{\Bigg/}}
\newcommand{\sij}[3]{\viiva\limits_{\hspace*{-5mm}{#1}}^{\hspace*{5mm}{#2}}{#3}}
\newcommand{\R}{\mathbb{R}}
\newcommand{\N}{\mathbb{N}}
\newcommand{\B}[1]{\mathbf{#1}}
\newtheorem{theorem}{Theorem}
\newtheorem{corollary}{Corollary}
\newtheorem{lemma}[theorem]{Lemma}
\newtheorem{prop}[theorem]{Proposition}
\newtheorem*{remark}{Remark}

\def\BibTeX{{\rm B\kern-.05em{\sc i\kern-.025em b}\kern-.08em
    T\kern-.1667em\lower.7ex\hbox{E}\kern-.125emX}}
\begin{document}

\title{Stochastic geometry analysis of a narrow-beam LEO uplink with mixed Gaussian shadowing}
\author{IEEE Publication Technology,~\IEEEmembership{Staff,~IEEE,}
        % <-this % stops a space
\thanks{This paper was produced by the IEEE Publication Technology Group. They are in Piscataway, NJ.}% <-this % stops a space
\thanks{Manuscript received October 8, 2023; revised December 8, 2023.}}

% The paper headers
\markboth{Journal of \LaTeX\ Class Files,~Vol.~1, No.~2, December~2023}%
{Shell \MakeLowercase{\textit{et al.}}: A Sample Article Using IEEEtran.cls for IEEE Journals}

\IEEEpubid{}
% Remember, if you use this you must call \IEEEpubidadjcol in the second
% column for its text to clear the IEEEpubid mark.


\maketitle
\begin{abstract}
  This paper presents an in-depth analysis of the joint probability distributions of the Signal-to-Interference Ratio (SIR) and Signal-to-Interference-plus-Noise Ratio (SINR) in a narrow-beam Low Earth Orbit (LEO) satellite uplink system with interference cancellation techniques. The increasing deployment of LEO satellite constellations for global communication necessitates a thorough understanding of interference management and signal quality metrics to ensure reliable and efficient data transmission. We develop a comprehensive analytical framework to model the impact of narrow-beam antennas on both SIR and SINR, accounting for the critical factors influencing signal propagation and interference under LEO dynamics. Additionally, we explore various interference cancellation strategies to enhance system performance. By deriving closed-form expressions and conducting extensive simulations, we quantify the joint probabilities for different system parameters, providing valuable insights into the trade-offs and benefits of interference cancellation in these systems. Our findings offer practical guidelines for optimizing LEO uplink communication strategies, thereby contributing to the advancement of high-capacity and low-latency satellite communication networks. (CHAT GPT)
\end{abstract}

% Note that keywords are not normally used for peerreview papers.
\begin{IEEEkeywords}
  LEO, SIR meta distribution, Nakagami fading
\end{IEEEkeywords}


\section{Introduction}

In recent years, the deployment of Low Earth Orbit (LEO) satellite constellations has gained significant momentum due to their potential to provide global, high-speed, low-latency communication services. As the demand for such capabilities continues to rise, it becomes increasingly crucial to address the challenges associated with managing interference and maintaining robust signal quality within these systems. One of the primary metrics for assessing signal quality in communication systems is the Signal-to-Interference Ratio (SIR) and Signal-to-Interference-plus-Noise Ratio (SINR). These metrics become even more pertinent in LEO uplink scenarios, where multiple users concurrently transmit signals to satellites with limited frequency resources.

The narrow-beam antenna technology, which focuses transmission power into a smaller beamwidth, offers a promising approach to enhance signal quality and reduce interference. However, the dynamic nature of LEO satellites, coupled with the high density of ground terminals, creates complex interference scenarios that necessitate advanced analytical tools and methodologies for effective management. In this context, understanding the joint probability distributions of SIR and SINR becomes essential for designing robust LEO uplink systems.

This paper addresses this critical need by developing a comprehensive analytical framework to model the joint probability distributions of SIR and SINR in a narrow-beam LEO uplink environment. We incorporate various system parameters, such as satellite altitude, beamwidth, user density, and interference sources, to capture the intricate nature of signal propagation and interference in LEO systems. Furthermore, we investigate the effectiveness of different interference cancellation techniques, which are pivotal in mitigating interference and enhancing communication performance.

By deriving closed-form expressions and performing extensive simulations, our study provides valuable insights into the impact of narrow-beam antennas and interference cancellation on the joint probabilities of SIR and SINR. The results of this research not only shed light on the fundamental trade-offs and potential benefits of these technologies but also offer practical guidelines for optimizing LEO uplink communication strategies. Ultimately, this work aims to contribute to the advancement of high-capacity, reliable, and efficient satellite communication networks, paving the way for the continued growth and success of global connectivity solutions.





\begin{table}
  \begin{center}
    \begin{tabular}{| c | p{4.5cm}  |p{1.5cm}|}
      \hline
      \multicolumn{3}{|c|}{Glossary of principal symbols} \\
      \hline
      Symbol& Explanation &Value
      \\ 
      \hline
      $h$ & Altitude of the SBSs.&$1200$ [km] \\
      $\epsilon$ & Elevation angle of the SBSs.& $(90 \degree,40 \degree)$ \\
      $G[\cdot]$ & The SBS antenna gain.&\\
      $\alpha$ &Power path loss exponent.& $4$\\
      $\varphi_{\text{RX}}$ & Width of the SBSs $3$ dB gain.& $1.6 \degree$  \\
      $\Theta \subset E $ & Poisson p.p. on the earth's surface $E \subset \R^3$.& \\
      $\Phi \subset \R^2$ & Poisson p.p. on the plane. &\\
      $x_0$ & Nearest point to the origin in $\Phi$.&  \\
      $\lambda$ & Density parameter of $\Phi$ and $\Theta$.& \\
      $\kappa$ & Parameter that reflects the approximate mean number of UEs inside a SBS's $3$ dB footprint;  $\kappa = h^2\pi \lambda \varphi_{\textup{RX}}^2/\sin^4(\epsilon)$.& \\
      ${\tilde{\kappa}}$ &  $\kappa/\log(2)$.&\\
      $H_{\text{log}}$ & Mixed log-normal shadowing r.v.&     \\
      $p_{\text{LoS}}$& LoS probability & $(0.99,0.61)$\\
      $p_{\text{NLoS}}$&  $1-p_{\text{LoS}}$ & $(0.01,0.39)$\\
      $\mu_{\text{LoS}}$& Shadowing mean of the LoS component & 0 [dB] \\
      $\mu_{\text{NLoS}}$& Mean of the NLoS component & -26 [dB] \\
      $\sigma_{\text{LoS}}$& Variance of the LoS component & 4 [dB] \\
      $\sigma_{\text{NLoS}}$& Variance of the NLoS component & 6 [dB] \\
      $p_{\text{NLoS}}$&  $1-p_{\text{LoS}}$ & $(0.01,0.39)$\\
      $H_{\text{exp}}$ & Defective exponential shadowing r.v.    & \\
      $\theta$ & SIR or SINR threshold for a successful transmission.&\\
      $I$ & Interference at the typical SBS in the plane model.&\\
      $S$ & The signal power of the served UE at the typical SBS in the plane model.&\\
      $\mathring{I}$ & Interference at the typical SBS in the spherical model.&\\
      $\mathring{S}$ & The signal power of the served UE at the typical SBS in the spherical model.&\\ 
      $\hat{d}_{h,\epsilon}$ & The distance between the SBS and the center of the footprint in the plane model.&\\
      $d_{0}$ & Normalizing distance. & \\                        
      \hline
    \end{tabular}
  \end{center}
\end{table}   


\section{Quantities of interest}
We consider the ``typical SBS'' with a narrow Gaussian beam. The UEs are located on the plane according to the PPP. The SBS antenna boresigh is directed at $\textit{o} \in (0,0) \in \R^2$. The planar model approximates the spherical model, where the UEs are located on a spherical Earth's surface. We will come back to the spherical model in the numerical results section.



\subsection{Gain process}
The path loss is Gaussian, a exponentially decaying function, representing the antenna gain of a narrow-beam LEO satellite. The antenna gain loss is a function of the distance $r$ from the origin $\textit{o}\in \R^2$, 
\begin{equation}
  G(r) = 2^{-r^2/\varphi^2_{\text{RX}}},
\end{equation}
where $\varphi_{\text{RX}}$ is the width of the $3$ dB gain and $D_{h,\epsilon}=\sin^2(\epsilon)/h$ is a scaling constant. 

Given a homogeneous PPP $\Phi \subset \R^2$ of density $\lambda$ and i.i.d. fading variables $\{H_x\}$, we define the process of the perceived gains at the receiver; the gain process (GP):
\begin{equation}
  \label{eq:gainprocess}
  \mathcal{G} \triangleq \left\{ H_x G(D_{h,\epsilon}\|x\|):x \in \Phi  \right\}.
\end{equation}
The GP is a \textit{projection process} and, as such, a nonhomogeneous PPP [CITE]. The density of the GP has the following connection to the fading distribution.

\begin{prop}[Density of the GP]
  Let $F_H(\cdot)$ be the complementary cdf (ccdf) of $H$. The density of $\mathcal{G}$ is given by
  \begin{equation}
    \label{eq:GPdensity}
    \lambda_{\mathcal{G}}(t)= \tilde{\kappa}F_H(t)/t, \text{ \text{for} }t>0,
  \end{equation}
  where $\tilde{\kappa}= \kappa/\log(2)$ and
  \begin{equation}
    \label{eq:kappa}
          {\kappa} \triangleq    \pi \lambda   \left(\frac{\varphi_{\textup{RX}}}{D_{h,\epsilon}}\right)^2,
  \end{equation}
  where $D_{h,\epsilon}=\sin^2(\epsilon)/h$. The parameter $\kappa$ has an interpretation as the mean number of UEs inside the $3$ dB footprint of a SBS (see appendix XX).
  \begin{proof}
    Let $f_H(\cdot)$ be the pdf of $H$. We denote $G^{-1}(\cdot)$ as the generalized inverse $G^{-1}(y) = \inf \{x:G(x)<y\}$. By [CITE],
    \begin{align*}
      &\int_t^{\infty}\lambda_{\mathcal{G}}(y)dy \overset{}{=} \pi \lambda \mathbb{E}\left[ \left(\frac{G^{-1}(t/H)}{D_{h,\epsilon}}\right)^2 \right] \\
      &=\pi \lambda \int_t^{\infty} \left(-\frac{\varphi_{\text{RX}}\sqrt{-\log(t/h)}}{D_{h,\epsilon} \sqrt{\log(2)}}\right)^2f_H(h) dh  \\
      &= -\tilde{\kappa} \int_t^{\infty} \log(t/h)f_H(h) dh\\
      &\overset{(a)}{=} -\tilde{\kappa} \sij{t}{\infty} \log(t/h)F_H(h) - \tilde{\kappa} \int_t^{\infty}\frac{F_H(h)}{h} dh.
    \end{align*}
    In (a), we use the integration by parts. The result follows by derivating w.r.t. $t$ and taking the minus sign---as long as $\int_t^{\infty} \log(t/h)f_H(h) dh$ converges for all $t>0$.
  \end{proof}
\end{prop}
In this work, we use the value $\varphi_{\text{RX}} = 1.6 \degree$ according to the International Telecommunication Union Recommendations (ITU-R) \cite{ITURS1528}---furthermore, we fix $h=1200$ km. However, note that $\kappa$ is the essential spatial system parameter, and other parameters only have a minor impact on the simulated values in the spherical model.




The mean and the variance of the \textit{total received power}
\begin{equation}
  \label{eq:totpow}
  I \triangleq \sum\limits_{u \in \mathcal{G}} u = \sum\limits_{x \in \Phi} H_x G(D_{h,\epsilon}\|x\|)
\end{equation}
  are given by
\begin{align}
  \label{eq:totmean}
  &\mathbb{E}\left(I \right) = \int_{0}^{\infty} t\lambda_{\mathcal{G}}(t) dt = \tilde{\kappa} \int_{0}^{\infty}F_H(t) dt \nonumber \\
  &=\tilde{\kappa} \mathbb{E}(H), \\\
  \label{eq:totvar}
  &\text{var}\left(I \right) = \int_{0}^{\infty} t^2\lambda_{\mathcal{G}}(t) dt= \tilde{\kappa} \int_0^{\infty}tF_H(t) dt  \nonumber \\
  &= \tilde{\kappa} \frac{\text{var}(H) + \mathbb{E}(H)^2}{2} = \tilde{\kappa}  \mathbb{E}[H^2]/2,
\end{align}
respectively.

The spatial path loss is of no interest here: because we assume a narrow beam $G(\cdot)$, the relevant UEs are located close to each other; hence, the spatial path loss is only a constant, and $\mathbb{E}(I)$ and $\text{var}(I)$ can be scaled accordingly. Furthermore, the spatial path loss does not affect the SIR. However, it is taken into account in the SINR analysis. Further, the signal powers are incorporated into the noise.




\subsection{Shadowing}
Unfortunately, unlike with a singular path loss, when the density of the projection process depends only on a single moment of $H$ regardless of the distribution, $\lambda_{\mathcal{G}}(t)$ has a pointwise dependence on $F_H(t)$. However, for analytical tractability, we can use the defective exponential power fading distribution
\begin{equation}
  \label{eq:defexp}
  F_{{H}_{\text{exp}}}(t)=\rho_{\epsilon} e^{- t},
\end{equation}
where $0<\rho_{\epsilon} \leq 1$ reflects the probability that the shadowed signal power is above $0$. The atomic probability measure $1-\rho_{\epsilon}$ at $0$ reflects the fading occurrences when the signal is blocked entirely. By matching the first two moments (this is equivalent to matching the mean \eqref{eq:totmean} and the variance \eqref{eq:totvar} of the total received power), the defective exponential distribution effectively approximates the log-normal shadowing if the fading is severe with high variance. As implied in the lower index, the constant $\rho_{\epsilon}$ depends on the elevation angle $\epsilon$ because of the varying LoS probability that affects the shadowing distribution.

         \begin{figure}[h]
           \centering
           \includegraphics[width=\linewidth]{plotdensities.pdf}
           \caption{The density function of the GP for a two-tier mixed log-normal shadowing distribution with the corresponding exponential shadowing acquired by matching the first two moments. The parameters $\rho_{\epsilon} \in \{0.85,0.53\}$ for the elevation angles $\epsilon \in \{90^{\circ},40^{\circ}\}$, respectively. The no-shadowing $H\equiv 1$ is also plotted for comparison.} 
           \label{fig:plotdensities}
         \end{figure}

%% \subsection{Slow-fading distribution}
%% While analytically tractable, the defective exponential distribution can capture the mean and the variance of a more complicated fading distribution with high variance, particularly the lognormal distribution, which is a well-established shadowing model in the LEO networks. The shadowing is defined by the mixed exponential RV
%% \begin{equation}
%%   \hat{H}_x \sim
%%   \begin{cases}
%%     0, \text{ if } U < 1-\rho,\\
%%     \text{Exp}(\mu), \text{ if } U \geq1- \rho,              \label{eq:tier2exponential}
%%   \end{cases}
%% \end{equation}
%% with $\mu>0$ and $0<\rho\leq1$, and $U \sim U(0,1)$ follows the uniform distribution.


%%where
%% \begin{align}
%%   &a=\frac{2 (p_{\text{LoS}}-1) e^{{\mu_{N\text{LoS}}}+\frac{{\sigma_{N\text{LoS}}}^2}{2}}-2 p_{\text{LoS}} e^{{\mu_{\text{LoS}}}+\frac{{\sigma_{\text{LoS}}}^2}{2}}}{(p_{\text{LoS}}-1) e^{2 \left({\mu_{N\text{LoS}}}+{\sigma_{N\text{LoS}}}^2\right)}-p_{\text{LoS}} e^{2 \left({\mu_{\text{LoS}}}+{\sigma_{\text{LoS}}}^2\right)}},\\
%%   &b=\frac{2 \left(({p_\text{LoS}}-1) e^{{\mu_{\text{N\text{LoS}}}}+\frac{{\sigma_{\text{N\text{LoS}}}}^2}{2}}-{p_\text{LoS}} e^{{\mu_{\text{LoS}}}+\frac{{\sigma_{\text{LoS}}}^2}{2}}\right)^2}{({p_\text{LoS}}-1) e^{2 \left({\mu_{\text{N\text{LoS}}}}+{\sigma_{\text{N\text{LoS}}}}^2\right)}-{p_\text{LoS}} e^{2 \left({\mu_{\text{LoS}}}+{\sigma_{\text{LoS}}}^2\right)}}.              
%% \end{align} 



\subsection{Laplace transform of the total received power}

With exponential shadowing ${H}_{\text{exp}}$, for Re$(s)>1$,

\begin{align}
  \label{eq:lapdef}
  &\mathcal{L}_{I}(s)\triangleq \mathbb{E}\left(e^{-sI}\right)= \exp\left\{-\int_0^{\infty}(1-e^{-sr}) \lambda_{\mathcal{G}}(r) dr \right\} \nonumber \\
  &=\exp\left\{-\tilde{\kappa}\int_0^{\infty}(1-e^{-sr}) F_{{H}_{\text{exp}}}(r) /r dr \right\} \nonumber \\
  &=\exp\left\{-\tilde{\kappa}\rho_{\epsilon}\int_0^{\infty}(1-e^{-sr}) e^{-r} /r dr \right\} \nonumber \\
  &=(1+s)^{-\tilde{\kappa}\rho_{\epsilon}},
\end{align}
which is the Laplace transform of the gamma distribution with the shape parameter $\tilde{\kappa}\rho_{\epsilon}$. This characterization of $I$ is a key insight in the analysis of the joint SIR distribution: $I$ can be interpreted as a gamma subordinator (gamma process) at time $\tau=\tilde{\kappa} \rho_{\epsilon}$, and the density of the factorial moment measure of the SIR of the first $n$ UEs follows.


\subsection{SINR and STINR processes and their factorial moment measures}
Let $N \geq 0$ be the noise power constant. We denote the signal-to-interference-plus-noise ratio (SINR) process of the UEs as
\begin{align}
  \label{eq:STIR}
  \Psi &= \{\mathsf{Z}: \mathsf{Z} \in \Psi\} \triangleq \left\{ \frac{u}{N+I-u} : u \in \mathcal{G}\right\} \\
  &=\left\{ \frac{H_x G(D_{h,\epsilon}\|x\|)}{N+I-H_x G(D_{h,\epsilon}\|x\|)} : x \in \Phi\right\},
\end{align}
where $I$ is defined in \eqref{eq:totpow}. Similarly, the signal-to-total-interference-plus-noise (STINR) process is defined as
\begin{align}
  \label{eq:STINR}
  \Psi' &= \{\mathsf{Z}': \mathsf{Z}' \in \Psi'\} \triangleq \left\{ \frac{u}{N+I} : u \in \mathcal{G}\right\}.
\end{align}
We can always recover on process from another
\begin{equation}
  \Psi = \left\{ \frac{\mathsf{Z}'}{1- \mathsf{Z}'}: \mathsf{Z}' \in \Psi' \right\}, \hspace{0.3cm} \Psi' = \left\{ \frac{\mathsf{Z}}{1+ \mathsf{Z}}: \mathsf{Z} \in \Psi \right\}.
\end{equation}
Let $\theta$ denote the SINR threshold of successful transmission. The event $\Psi \ni\mathsf{Z}> \theta$ is equivalent to $\Psi' \ni \mathsf{Z}'> \theta'$  with $\theta' \triangleq \theta/(1+\theta)$ and $\theta \triangleq \theta'/(1-\theta')$.

The factorial moment measure of the SINR process is defined on the sets $(t_1,\infty] \times \dots \times (t_n, \infty]$ with $t_1,\dots,t_n\geq 0$
    \begin{align}
      M^{(n)}(t_1,\dots,t_n) &\triangleq M^{(n)}((t_1,\infty],\dots,(t_n,\infty]) \nonumber \\
          & \triangleq \mathbb{E} \left( \sum^{\text{distinct}}_{\left(\mathsf{Z}_1, \dots, \mathsf{Z}_n \right) \in (\Psi)^{\times n}} \prod_{i=1}^n \mathds{1}(\mathsf{Z}_i >t_i)\right),
    \end{align}
    where $\mathds{1}(\cdot)$ is the indicator function. Similarly, for the STINR process,
    \begin{align}
          M'^{(n)}(t'_1,\dots,t'_n) &\triangleq M'^{(n)}((t'_1,\infty],\dots,(t'_n,\infty]) \nonumber \\
              &\triangleq \mathbb{E} \left( \sum^{\text{distinct}}_{\left(\mathsf{Z}'_1, \dots, \mathsf{Z}'_n \right) \in (\Psi')^{\times n}} \prod_{i=1}^n \mathds{1}(\mathsf{Z}'_i >t'_i)\right).
    \end{align}
    ``Distinct'' indicates that $\mathsf{Z}_i \neq \mathsf{Z}_j $ for all $i \neq j $. The (partial) density is
    \begin{align}
      \label{eq:differatemomentmeasure}
     &{\mu'}_n^{(n+i)}(z'_1,\dots,z'_n)= (-1)^n \frac{\partial^n M'^{(n+i)}(t'_1\dots t_n', z'_n, \dots, z'_n)}{\partial t'_1 \dots \partial t'_n} \nonumber\\
      &\hspace{4cm}|(t'_1=z_1'\dots t'_k=z'_n),
    \end{align}
    for $\sum_{i=1}^k z'_i+iz'_k \leq 1$ and $0$ otherwise.

    If $i > 0$, \eqref{eq:differatemomentmeasure} is called the partial density of $M'^{(n)}$. 

    The density of the factorial \textit{n}th moment measure of the SINR process can be extracted as
    \begin{align}
      \label{eq:densitySINR}
      &\mu^{(n)}(z_1,\dots,z_n) \nonumber\\
      &= \prod_{j=1}^n\frac{1}{(1+z_j)^2}\mu'^{(n)}\left(\frac{z_1}{1+z_1},\dots,\frac{z_n}{1+z_n}\right).
    \end{align}

    
  
\subsection{Order statistics of the STINR process}
We denote $\mathsf{Z}'_{(1)}>\mathsf{Z}'_{(2)} >\mathsf{Z}'_{(3)} \dots$ the order statistics of the STINR process $\Psi'$, such that $\mathsf{Z}'_{(1)}$ is the larges value in $\Psi'$.


The joint pdf of the vector of $k$ strongest values of the STIR process $(\mathsf{Z}'_{(1)}, \dots, \mathsf{Z}'_{(n)})$ is given as a series expansion involving the partial densities
\begin{equation}
  \label{eq:jointprobability}
  f'_{(k)}(z'_1,\dots,z'_k)= \sum^{i_{\text{max}}}_{i=0}\frac{(-1)^i}{i!}{\mu'}_k^{(k+i)}(z'_1,\dots,z'_k),
\end{equation}
for $z'_1>z'_2>\dots>z'_k$ and $f'_{(k)}(z'_1,\dots,z'_k) =0 $ otherwise. The upper bound for the index $i_{\text{max}}<1/z'_k-k$ is the non-zero terms of the series expansion.

The $k$-coverage probability that the first $k$ strongest signals reach the threshold $\tau' =\tau/(1+\tau)$ is given by
\begin{align}
  \label{eq:kprobability}
  &\mathcal{P}^{(k)}(\theta) \triangleq  \int_{\theta'}^1\dots \int_{\theta'}^1 f'_{(k)}({z'_1},\dots,{z'_k})dz'_1 \dots d{z'_k}, \nonumber\\
  &=\sum_{i=0}^{i_{\text{max}}}\frac{(-1)^i}{i!}\int_{\theta'}^1\dots \int_{\theta'}^1 {\mu'}_k^{(k+i)}(z'_1,\dots,z'_k) \nonumber\\
  &\hspace{3.4cm}\times\mathds{1}(z'_1>\dots>z'_k) dz'_1 \dots d{z'_k}
\end{align}
with $\theta'=\theta/(1+\theta)$ and $i_{\text{max}}<1/\theta'-k$.


We still need to explicitly determine the expressions for the partial density ${\mu'}_n^{(n+i)}$ of the factorial moment measure $M^{(n)}$. We will tackle this separately for the interference-only channel and the interference-plus-noise-limited channel. For the interference-only channel, the results can be obtained from those of the interference-plus-noise-limited channel by setting the noise to $0$. However, the latter scenario is numerically more demanding.


\subsection{Partial density of the factorial moment measure in the interference-limited channel}
In this section, we characterize $\mu_n'^{(n+i)}$ of the STIR process, \textit{i.e.}, the STINR process with $N=0$, and the order statistics through the Poisson-Dirichlet distribution PD(0,$\rho_{\epsilon}$). 

Let $(X_{\tau}, \tau\geq 0)$ be a \textit{gamma subordinator}, or the \textit{gamma process}. It is a pure-jump increasing Lévy process with the intensity measure $\lambda(r) = e^{- r}/r$ . Assume that $(X_{\tau})$ has no drift component;
\begin{equation}
  \label{eq:lapsubord}
  \mathbb{E}(\exp\{-s X_{\tau}\}) = \exp\left\{-{\tau} \int_0^{\infty}(1-e^{-s r})\frac{e^{-r}}{r} dr \right\}.
\end{equation}
But \eqref{eq:lapsubord} equals to \eqref{eq:lapdef} at $\tau=\tilde{\kappa}\rho_{\epsilon} $. Let $V_1(X_{\tau}) \geq V_2(X_{\tau})\geq \dots \geq 0 $ denote the jumps of the subordinator $(X_{\tau})$ at time $\tau$. Recall the STIR process with $N=0$ \eqref{eq:STINR}. By definition, the sequence
\begin{equation}
  \label{eq:relativesequence}
  \left(\frac{V_1(X_{\tilde{\kappa}\rho_{\epsilon}})}{X_{\tilde{\kappa}\rho_{\epsilon}}},\frac{V_2(X_{\tilde{\kappa}\rho_{\epsilon}})}{X_{\tilde{\kappa}\rho_{\epsilon}}} \dots \right)
\end{equation}
is the ordered sequence $(\mathsf{Z}'_{(1)},\mathsf{Z}'_{(2)} \dots)$. It has the Poisson-Dirichlet distribution PD$(0, \rho_{\epsilon} \tilde{\kappa})$. 




\begin{prop}
  The density of the n\textit{th} factorial moment measure of the STIR process at the NB LEO with the Gaussian antenna beam is given by
  \begin{align}
    \label{eq:factorialmoment}
    \mu'^{(n)}(t_1',\dots,t'_n) = (\tilde{\kappa}b)^n\prod_{j=1}^n{t'}_{j}^{-1}\left(1- \sum_{j=1}^nt'_j \right)^{\tilde{\kappa}b-1},       
  \end{align}
  whenever $t_1>\dots >t_n$ and $\sum_{i=1}^n t_i <1$, and $0$ otherwise.
\end{prop}
To study the joint distribution of order statistics of the STINR process, we define some auxillary functions. We write for $i\geq 1$
\begin{align}
  \label{eq:auxillary}
  &{\mu'}_k^{(k+i)}(z'_1,\dots,z'_k) \nonumber \\
  &= \int_{z'_k}^1 \dots \int_{z'_k}^1 {\mu'}^{(k+i)}(z'_1,\dots,z'_k,\zeta'_1,\dots,\zeta'_i) d\zeta'_1 \dots \zeta'_i.
\end{align}
Recall the condition for the density being non-zero for $\sum_{i=1}^kz'_k+iz_k \leq 1 $.








\subsection{Partial density of the factorial moment measure in the interference-plus-noise-limited channel}


In the interference-plus-noise-limited channel, we derive the partial densities through differentiating the factorial moment measure.

\begin{prop}
  The $n$-moment measure of the STINR process is given by

  \begin{align}
    \label{eq:nthmomentmeasureSTINR}
      &M'^{(n)}(t'_1,\dots,t'_n) =\left(\tilde{\kappa} \rho_{\epsilon}\right)^n \nonumber\\
      &\times \int_1^{\infty}\dots\int_1^{\infty} e^{-N\hat{T}_n\sum\limits_{i=1}^nt'_iv_i} \left(1+\hat{T}_n\sum\limits_{i=1}^nt'_iv_i\right)^{-\tilde{\kappa}b} \nonumber\\
      &\times \frac{\prod\limits_{i=1}^nt'_i}{\sum\limits_{i=1}^nt'_iv_i}  \left[\sum\limits_{j=1}^n\frac{1}{\prod\limits_{i\neq j}\left(t'_i v_i+ t_i'\hat{T}_n\sum\limits_{k=1}^nt'_kv_k \right)} \right] dv_1 \dots dv_n,
    \end{align}
    where $\hat{T}_n= 1/(1-\sum_{i=1}^nt'_i)$.


\end{prop}
 
The partial densities are given by
    \begin{align}
      \label{eq:differatemomentmeasure}
     &{\mu'}_n^{(n+i)}(z'_1,\dots,z'_n)= (-1)^n \frac{\partial^n M'^{(n+i)}(t'_1\dots t_n', z'_n, \dots, z'_n)}{\partial t'_1 \dots \partial t'_n} \nonumber\\
      &\hspace{4cm}|(t'_1=z_1'\dots t'_n=z'_n)
    \end{align}
The differentiation of the integrand in \eqref{eq:nthmomentmeasureSTINR} is a tedious task, although straightforward. It can be done with symbolic tools such as Mathematica. We omit explicit representations of the differentiated integrands of $M'^{(n+i)}$.



\subsection{SIR and SINR under interference cancellation}


Let $(u_{(1)},\dots,u_{(k)} )\subset \mathcal{G}$ be an ordered set of points in the GP; $u_{(1)}$ denotes the strongest signal at the SBS. In the following, we consider the SINR of the n\textit{th} strongest signal, $n \leq k$, with the strongest $k$ signals canceled from the interference. We define the SINR under the interference cancellation (IC-SINR) as
\begin{equation}
  \text{SINR}_{n,k} \triangleq \frac{u_{(n)}}{N+I-\sum_{j =1 }^k u_{(j)}}.
\end{equation}


The IC-SINR can be formulated in terms of the STINR process:

%% \begin{align}
%%   \label{eq:IC-SINRcond}
%%   \mathcal{P}_{\text{IC}}^{(n,k)}(\theta) &\triangleq \mathbb{P}\{\text{SINR}_{n,k} > \theta \} \nonumber \\
%%   &=\mathbb{P} \left\{ u_{(n)} >\theta\left(N+I-  \sum_{j=1}^ku_{(j)}\right)\right\} \nonumber\\
%%   &\overset{}{=}\mathbb{P} \left\{(1+\theta) \frac{u_{(n)}}{N+1}+ \theta \frac{\sum^{j\neq n}_{j\in\{1,\dots,k\}} u_{(j)}}{N+I}>\theta \right\} \nonumber \\
%%   &\overset{(a)}{=} \mathbb{P} \left\{ \mathsf{Z}'_{(n)}+\theta'\sum^{j\neq n}_{j\in\{1,\dots,k\}}\mathsf{Z}'_{(j)} +>\theta'\right\}.
%% \end{align}

\begin{align}
  \label{eq:IC-SINRcond}
   & \mathds{1}\{\text{SINR}_{n,k} > \theta \} = \mathds{1} \left\{ u_{(n)} >\theta\left(N+I-  \sum_{j=1}^ku_{(j)}\right)\right\} \nonumber\\
  &\overset{}{=}\mathds{1} \left\{(1+\theta) \frac{u_{(n)}}{N+1}+ \theta \frac{\sum^{j\neq n}_{j\in\{1,\dots,k\}} u_{(j)}}{N+I}>\theta \right\} \nonumber \\
  &\overset{(a)}{=} \mathds{1} \left\{ \mathsf{Z}'_{(n)}+\theta'\sum^{j\neq n}_{j\in\{1,\dots,k\}}\mathsf{Z}'_{(j)} +>\theta'\right\}.
\end{align}


In (a), we divide both sides by $(1+\theta)$, which leads to the final result by the equivalence $\theta' = \theta/(1+\theta)$. For the interference-only channel, $N=0$, $\text{SIR}_{n,k} \triangleq \text{SINR}_{n,k}$




In independent interference cancellation (IIC), we cancel and decode all signals independently: if the first $k$ signals exceed the SINR threshold $\tau$, \textit{i.e.}, $\mathsf{Z}_{(j)} > \tau$ for all $j \in \{1,\dots,k\}$, we can subtract the $k$ strongest signals from the interference. An improvement for the IIC is the successive interference cancellation (SIC). First, we cancel and decode the strongest signal, then decode the second strongest signal by subtracting the strongest and second strongest signal from the interference, and so on. The SIC of the $k$ strongest signals requires a superposition of conditions
\begin{equation}
  \label{eq:SIC-SINRcond}
  \begin{cases}
    &\mathds{1} \left(\mathsf{Z}'_{(1)} > \tau'\right)\\
    &\mathds{1}\left( \mathsf{Z}'_{(2)} + \tau' \mathsf{Z}'_{(1)}> \tau'\right) \\
    &\hspace{1cm}\dots \\
    &\mathds{1} \left(\mathsf{Z}'_{(k)} + \tau' \sum_{j=1}^{k-1}\mathsf{Z}'_{(j)}> \tau'\right).
  \end{cases}
\end{equation}


Let us consider that the strongest signal is decoded directly if its SINR exceeds $\theta$; otherwise, signal cancellation is attempted. Accordingly, we define a IIC-SINR$_{k}$ coverage probability of the strongest signal with independent signal cancellation of the strongest $k$ signals with the signal separation threshold $\tau=\tau'/(1-\tau')$ 

\begin{equation}
  \mathcal{P}^{(k)}_{\text{IIC}}(\theta,\tau) \triangleq \mathbb{P}\begin{Bmatrix} \text{SINR}_{1,k}(\theta)>\theta & \text{if } \mathsf{Z}_{(k)}>\tau, \\
\mathsf{Z}_{(1)}>\theta & \text{otherwise} \end{Bmatrix}.
\end{equation}
Similarly, SIC-SINR$_{k}$ coverage probability is defined by
\begin{equation}
  \label{eq:SIC-SINRprob}
  \mathcal{P}^{(k)}_{\text{SIC}}(\theta,\tau) \triangleq \mathbb{P}\begin{Bmatrix} \text{SINR}_{1,k}(\theta)>\theta & \text{if for all }n\in\{1,\dots,k\}  \\
    & \mathsf{Z}'_{(n)}+\tau'\sum\limits_{j=1}^{n-1}\mathsf{Z}'_{(j)}>\tau',\\
\mathsf{Z}_{(1)}>\theta & \text{otherwise} \end{Bmatrix}.
\end{equation}
We have that $\mathcal{P}^{(k)}_{\text{SIC}}(\theta,\tau)\geq \mathcal{P}^{(k)}_{\text{IIC}}(\theta,\tau) \geq \mathcal{P}^{(1,1)}(\theta)$ for all $0<\tau \leq \theta$. 


 As a metric of the performance improvement, we study the difference between the coverage probability without any interference cancellation and the IIC or SIC. 
\begin{align}
  \Delta^{(k)}_{\text{IIC}}(\theta,\tau) \triangleq \mathcal{P}^{(k)}_{\text{IIC}}(\theta,\tau)- \mathcal{P}^{(1)}(\theta),\nonumber\\
  \Delta^{(k)}_{\text{SIC}}(\theta,\tau) \triangleq \mathcal{P}^{(k)}_{\text{SIC}}(\theta,\tau)- \mathcal{P}^{(1)}(\theta),
\end{align}
where the $1$-coverage probability  $\mathcal{P}^{(1)}(\theta)$ is given in $\eqref{eq:kprobability}$. Using the joint pdf of the STIR process \eqref{eq:jointprobability}, we have for the SIC-SINR
\begin{prop}
  \begin{align}
    \label{eq:SICprob}
    & \Delta^{(k)}_{\text{SIC}}(\theta,\tau) \nonumber\\
    &=\sum^{i_{\text{max}}}_{i=0} \int_{0}^{\theta'} \dots \int_{0}^{\theta'}\frac{(-1)^i}{i!}\nonumber \\
    & \hspace{0.5cm}\times \prod_{n=1}^k\mathds{1}\left( z'_n+ \tau'\sum_{j=1}^{n-1} z'_j>\tau' \right)  \mathds{1}\left(z_1+  \theta'\sum_{j=2}^k z'_j>\theta' \right) \nonumber \\
    &\hspace{0.5cm} \times \mathds{1}(z'_1>\dots>z'_k){\mu'}_k^{(k+i)}(z'_1,\dots,z'_k) d z'_1 \dots d z'_k,
  \end{align}
  with the upper summation limit bounded by $i_{\text{max}} < 1/\tau'-1=1/\tau.$ 
  \begin{proof}
    The expression follows from the joint pdf of the order statistics \eqref{eq:jointprobability} and the conditions \eqref{eq:IC-SINRcond} and \eqref{eq:SIC-SINRcond}. We relaxed the upper bound of  $i_{\text{max}}$ to $1/\tau'-1$ (c.f. the upper limit given for \eqref{eq:jointprobability}; $i_{\text{max}} < 1/z'_{(k)}-k‰$ in which the number of the summation terms would be approaching infinity near the $k$\textit{th} dimension lower limit of the integrand.) Luckily, the LHS conditioning of the integrand limits the non-zero terms of the partial density series expansion. Namely, we have that $z'_{(n)}+\tau'\sum_{j=1}^{n-1}z'_{(j)}>\tau'$. By simple algebra, $\sum_{j=1}^{n-1}z_{(j)}> 1-z_{(n)}/\tau'$. Recall the condition on the non-zero terms of the density of $M'^{(n+i)}$:  $\sum_{j=1}^n z'_{(j)}+i z'_{(n)} =\sum_{j=1}^{n-1}z'_{(j)} +z'_{(n)}+i z'_{(n)}  \leq 1$. The condition certainly \textit{does not} hold if $1-z_{(n)}/\tau'+ z'_{(n)}+i z'_{(n)}>1$. We arrive at the inequality $z'_{(n)} \left(-1/\tau' + 1 +i \right)>0$. Divide both sides by $z'_{(n)}$, and the general upper bound of $i$ follows.
  \end{proof}
\end{prop}
The SINR-IIC$_{(k)}$ case $\Delta^{(k)}_{\text{IIC}}(\theta,\tau)$ can be obtained by dropping the summation term inside the LHS condition in $\eqref{eq:SICprob}$, essentially bounding the lower integration limit to $\tau'$.



\subsubsection{Density function of the SIR of the strongest signal}
Combining \eqref{eq:densitySINR}, \eqref{eq:jointprobability}, and \eqref{eq:factorialmoment}, we get a closed form for the pdf of SIR$_{1,1}$ in the region $z\geq 1$:


\begin{equation}
  \label{eq:SIR1}
  f_{(1)}(z) = \frac {\tilde{\kappa}\rho_{\epsilon}\left({z + 1} \right)^{-\tilde{\kappa}\rho_{\epsilon}}} {z}.
\end{equation}



We may calculate a lower bounds for the first two moments for the SIR:


\begin{align}
  &\mathbb{E}(\text{SIR}_{1,1})  \geq\int_{1}^{\infty}f_{(1)}(z)zdz=\frac{\tilde{\kappa}\rho_{\epsilon}2^{1-\tilde{\kappa}\rho_{\epsilon} }}{\tilde{\kappa}\rho_{\epsilon}-1}, \tilde{\kappa}\rho_{\epsilon} >1, \\
  &\mathbb{E}(\text{SIR}^2_{1,1}) \geq \int_{1}^{\infty}f_{(1)}(z)z^2dz = \frac{(\tilde{\kappa}\rho_{\epsilon}) ^2 2^{1-\tilde{\kappa}\rho_{\epsilon} }}{(\tilde{\kappa}\rho_{\epsilon} - 2)  (\tilde{\kappa}\rho_{\epsilon} - 1)}, \nonumber\\
  &\hspace{6.5cm}\tilde{\kappa}\rho_{\epsilon}>2 .
\end{align}
The mean and the second moment diverge for $\tilde{\kappa}b\leq 1$ and $\tilde{\kappa}b\leq 2$, respectively. This implies that the variance of the SIR of the strongest signal is undefined for $\tilde{\kappa}b \leq 1$ and infinite for $1 <\tilde{\kappa}b \leq 2$. The insight reflects that the user experience of the link quality varies significantly over the SBSs in the density region $\tilde{\kappa} \rho_{\epsilon} \leq 2$. To achieve a consistent SBS performance, we have to make the network denser than what corresponds to $\tilde{\kappa} \rho_{\epsilon} =2$. Recall $\eqref{eq:kappa}: \kappa = \tilde{\kappa}\log(2)$ reflects the average number of UEs inside a $3$ dB footprint of a SBS.  Additionally, the density is influenced by shadowing characteristics through the parameter  $0<\rho_{\epsilon} \leq 1$. One way to improve user fairness is to through the IC, which can help maintain good average performance while reducing the variance in the SIR at the SBSs.

\subsubsection{Numerical results regarding the average and variance of the SIR without interference cancellation and using IIC and SIC}
We observed that for $\tau = -7$ [dB], \textit{ i.e.}, $\tau =10^{-7/10}\approx 0.2$, good choices are $k \in\{2,3\}$ for IIC and SIC, respectively. Hence, we use this signal separation threshold and the corresponding $k$s as a baseline.

For $\rho_{\epsilon} \tilde{\kappa}=2$, the expected SIR of the strongest signal without IC is given by
\begin{equation}
  \mathbb{E}(\text{SIR}_{1,1})= \int_{0}^{\infty}\mathcal{P}^{(1)}(y)dy \approx 1.4.
\end{equation}
  Recall, tahat the variance var$(\text{SIR}_{1,1})=\infty$, which might not be desirable in terms of user fairness.

  By increasing $\rho_{\epsilon} \tilde{\kappa}$ (modifying $\lambda,h,\epsilon$ or $\varphi_{\text{RX}}$), we can reduce the variance in the SIR while maintaining the performance. For example, $ \mathbb{E}$(IIC-SINR$_{2}) \approx \mathbb{E}$(SIC-SINR$_{3}) \approx 1.4$ for $\rho_{\epsilon} \tilde{\kappa}\in \{2.6,3.8\}$, respectively. In this case, IIC-SINR variance is calculated by
\begin{align}
  \text{var(IIC-SINR}_{2})&=\int_{0}^{\infty}y\mathcal{P}_{\text{IIC}}^{(2)}(y,\tau)dy \nonumber \\
  &\hspace{1cm}-\left(\int_{0}^{\infty}\mathcal{P}_{\text{IIC}}^{(2)}(y,\tau)dy\right)^2  \approx 2.6
\end{align}
And similarly, var(SIC-SINR$_{3}) = 2\int_{0}^{\infty}y\mathcal{P}_{\text{SIC}}^{(3)}(y,\tau)dy-$ $\left(\int_{0}^{\infty}\mathcal{P}_{\text{SIC}}^{(3)}(y,\tau)dy\right)^2 \approx 2.9$. The standard deviation in both cases is below $2$, which is a remarkable improvement in the consistency of the link quality. Note that the increase of $\tilde{\kappa}$ does not necessarily mean making the SBS constellation more dense: with interference cancellation, a single SBS might be able to serve multiple UEs with satisfactory SIR. However, we only consider the UE with the strongest signal in the IIC and SIC.

The integrations in the IIC and SIC were conducted by trapezoidal method by first evaluating the SINR-IIC and SINR-SIC the ccdf values (\textit{i.e.} coverage probabilities) pointwise for sufficiently many $y$.






\section{Numerical results and the connection to the satellite communications}




\subsection{Gaussian mixture model}
Let us consider a two-tier $\mathcal{T} = \{\text{LoS},\text{NLoS}\}$ logarithmic Gaussian mixture fading model with the mean and standard deviation $\mu_{i},\sigma_{i}$, $i \in \{\text{LoS},\text{NLoS} \}$, for the line-of-sight and the non-line-of-sight tiers, respectively. Consequently, The power fading RV of a transmitter $x \in \Phi$ is given by
\begin{equation}
  H_x \sim
  \begin{cases}
    \text{Lognormal}(\mu_{\text{LoS}},\sigma_{\text{LoS}}^2), \text{ if } U<p_{\text{LoS}} \\
    \text{Lognormal}(\mu_{\text{NLoS}},\sigma_{\text{NLoS}}^2)\text{ if } U \geq p_{\text{NLoS}},              \label{eq:tier2lognormal}
  \end{cases}
\end{equation}
where $U \sim U(0,1)$ follows the uniform distribution.







The mean of the lognormal RV is given by $ m_i=\mathbb{E}(H|T=i) = \exp \{\mu_i+\sigma_i^2/2\}$.


%% \begin{verbatim}
%% \begin{table}
%% \begin{center}
%% \caption{Filter design equations  ...}
%% \label{tab1}
%%     \begin{tabular}{| c | c | c |c|}
%%       \hline
%%       & $\mathcal{P}^{(1)}(\cdot)$& $\mathcal{P}^{(2)}_{\text{IC}}(\cdot)$& $\mathcal{P}^{(2)}_{\text{SC}}(\cdot)$\\
%%       \hline
%%       $\mathbb{E}(\cdot)$&$1.4$ & $1.4$ &$1.4$\\ 
%%       \hline
%%       $\tilde{\kappa}$& $2/\rho_{\epsilon} $ &$2.6/\rho_{\epsilon}$& $3.4/\rho_{\epsilon}$\\
%%       \hline
%%       Var$(\cdot)$& $\infty$ & $2.6$ &$2.3$\\
%%       \hline 
%%     \end{tabular}
%% \end{center}
%% \end{table}
%% \end{verbatim}


\appendices

\section{Factorial moment measure of the STINR process}
Let $\mu_M \triangleq \sum_{i=1}^nt'_i/G(D_{h,\epsilon}\|x_i\|)=\sum_{i=1}^nt'_i/G(D_{h,\epsilon} r_i)$.

\begin{align}
  &M'^{(n)}(t'_1,\dots,t'_n) =(2 \pi \rho_{\epsilon})^n\nonumber\\
  & \times\int\limits_{(\R_+)^n} \mathcal{L}_N(\mu_MT_n)\mathcal{L}_I(\mu_MT_n)\mathcal{L}_D(\mu_MT_n) r_1 dr_1 \dots r_n dr_n\nonumber \\
  &=(2 \pi \rho_{\epsilon} \lambda)^n \int\limits_{(\R_+)^n} \exp\left\{-NT_n\sum\limits_{i=1}^nt'_i2^{\left(\frac{D_{h,\epsilon}r_i}{\varphi_{\text{RX}}}\right)^2}\right\} \nonumber\\
  &\hspace{0.8cm}\times\left(1+T_n\sum\limits_{i=1}^nt'_i2^{\left(\frac{D_{h,\epsilon}r_i}{\varphi_{\text{RX}}}\right)^2}\right)^{- \tilde{\kappa}b} \frac{\prod\limits_{i=1}^nt'_i2^{\left(\frac{D_{h,\epsilon}r_i}{\varphi_{\text{RX}}}\right)^2}}{\sum\limits_{i=1}^nt'_i2^{\left(\frac{D_{h,\epsilon}r_i}{\varphi_{\text{RX}}}\right)^2}} \nonumber\\
  &\hspace{0.8cm}\times  \left[\sum\limits_{j=1}^n\frac{1}{\prod\limits_{i\neq j}\left(t'_i 2^{\left(\frac{D_{h,\epsilon}r_i}{\varphi_{\text{RX}}}\right)^2}+ t_i'T_n\sum\limits_{k=1}^nt'_k2^{\left(\frac{D_{h,\epsilon}r_k}{\varphi_{\text{RX}}}\right)^2} \right)} \right] \nonumber\\
  &\hspace{0.8cm}\times r_1 dr_1 \dots r_n dr_n.
\end{align}
The final results follows from the substitutions $\{u_i\}_{i= 1}^n$= $\{r_i D_{h,\epsilon}/\varphi_{\text{RX}}\}_{i= 1}^n$ and $\{v_i\}_{i= 1}^n$ =$\{ 2^{u^2_i}\}_{i= 1}^n$.





\begin{remark}
  Not all all two-tier log-normal shadowing scenarios can be characterized and approximated with a one-tier manner with the distribution \eqref{eq:defexp}. For example $\rho_{\epsilon}$ might be not within the region $0<\rho_{\epsilon} \leq 1$. This can occur, for example, in case of significant differences in the LoS or NLoS variances in the shadowing, like is the case in the suburban/??? sceneario in the 3GPP specs. We could approximate separately the LoS and NLoS components by the exponential function, leading to two i.i.d. gamma processes of the LoS and NLoS component transmitter. Treating such two separate gamma process's (representing the aggregate STIR) of the two separate shadowing tiers can be challenging, and will be left out of the scope of this paper. Anyhow, the approach presented in the paper works for the urban scenario with the 3GPP specs, because the two-tier log-normal shadowing can be in a reasonable manner approximated by a one-tier shadowing model using a signle defective exponential distribution.
\end{remark}


\bibliographystyle{IEEEtran}
%\bibliography{IEEEabrv, bib}
\bibliography{IEEEabrv,source}


\end{document}
